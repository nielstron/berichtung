\documentclass{article}
\RequirePackage[utf8]{inputenc}
\RequirePackage[T1]{fontenc}
\RequirePackage[ngerman]{babel} % TODO: Make this an option
\RequirePackage[babel,german=quotes]{csquotes}
\RequirePackage{hyperref}

\title{Ausflug in die Kohlegrube}
\author{Niels Mündler}
\date{\today}

\begin{document}
\maketitle

Die Exkursion in den Braunkohletagebau wurde mit großem Andrang erwartet.
Lediglich limitiert durch die Größe des Busses fuhr so etwa die halbe Akademie 14 Kilometer in den Süden Richtung Garzweiler.

Der Tagebau kündigte sich ab und an bereits weit vorher durch auf den Feldern verstreute kleine grüne Pumpen an.
Diese sorgen dafür, dass der Grundwasserspiegel von ca. 10 Meter auf die unter 230 Meter gesenkt wird, die der Tagebau Garzweiler tief ist.
Eine RWE-Mitarbeiterin wird uns später erzählen, dies habe keine Auswirkungen auf die umliegende Landwirtschaft, die weitreichende Folgen für Wasserqualität und grundwasserabhängige Natur wurden jedoch verschwiegen
\footnote{\url{https://www.bund-nrw.de/themen/mensch-umwelt/braunkohle/hintergruende-und-publikationen/braunkohle-und-umwelt/braunkohle-und-wasser/}}

Ein Tagebau, das ist ein Loch der Größe von 20 km$^2$ und der Tiefe von 350 Metern (mitunter auch tiefer und größer).
Es ist nicht statisch, auf einer Seite des Loches wird Gestein, eine Mischung aus Abraum und der begehrten Kohle,
abgetragen. Die Kohle wandert auf langen Förderbändern in Richtung eines nahegelegenen Kraftwerkes.
Ein weiter Transport lohnt sich energietechnisch nicht.
Der Abraum wird auf der anderen Seite an der Wand des Loches abgesetzt und mehrmals verdichtet.
Damit frisst sich die Öffnung des Tagebaus langsam und unaufhaltsam\footnote{
    Eine Ausnahme bilden hier Klimaaktivisten, die sich auf die Kohlebagger setzen, siehe \url{https://www.ende-gelaende.org/}
} durch die Nordrhein-Westfälische Landschaft.

Ohne große Ankündigung öffnet sich plötzlich zu unserer rechten diese gigantische Grube.
Gespickt mit kleinen Maschinen fahren wir entlang der Kohlegrube.
Zur Aussicht gehören hier: Kohlekraftwerke, mit Wasserdampfsäulen die aussehen als wären sie für die Wolken am Himmel verantwortlich.
Zehn Meter hohe Rechen, die umgekehrt im Boden stecken und eigentlich Sprühnebel erzeugen um aufsteigen Stäube zu binden.
Windräder, wild verstreut auf Hügeln, in nächster Nähe zu ihrem dicken Onkel der Energieerzeugungsfamilie.

Schließlich kommen wir am RWE Informationszentrum an.
Davor steht eine ausrangierte Kohleradschaufel wie eine Skulptur.
Weiter hinten wurden andere Schaufeln bunt angemalt und sollen für Kinder Spielmöglichkeiten bieten.
Wir werden in einen Kinosaal-artigen Raum geführt, erhalten jedoch keinen Film.
Stattdessen präsentiert uns eine Mitarbeiterin von RWE Folien über die Braunkohle.

Während des Vortrages, das sei angemerkt, ist jedoch keine klare Struktur zu erkennen.
Es scheint als stelle sich die Mitarbeiterin nur darauf ein, der Besuchsgruppe Paroli zu bieten und vergisst dabei
völlig Inhalt und Ziel ihres Vortrages.
Rückblickend soll nun wohl über die Relevanz von Braunkohle berichtet werden.

Braunkohle ist aktuell für knapp 40\% des Stromes in NRW verantwortlich und stellt circa 35\% des von RWE erzeugten Stromes dar
\footnote{Ein hervorragendes Tool für Statistiken bietet RWE selbst an: \url{http://rwe-kennzahlentool.de/}}.
Dies ist unter anderem der Grund dafür, dass Deutschland der größte CO2-Emittend in Europa ist.
Braunkohle sei dennoch wichtig, da sie die Schwankungen der erneuerbaren Energien ausgleichen soll.
Auch das Argument der \enquote{Verspargelung der Landschaft} durch die Windkraft fällt.
Es hat nicht die gleiche Zugkraft wenn es neben einem 30 km$^2$ großen Loch im Boden ausgesprochen wird und löst daher nur Gelächter aus.

Außerdem sei es die günstigste aller Energien.
Gleich nachgesetzt wird, dass der deutsche Staat den Strompreis sowieso viel zu hoch halte,
nur 20\% der Kosten von Braunkohlestrom stammen aus der Beschaffung, inklusive Umsiedlung und Renaturierung.
Weitere 20\% gehen an den Netzbetreiber und der Rest sei Steuern und Umlagen.
Der Steuerzahler werde damit viel zu stark belastet.
Da deshalb auch der deutsche Staat keine CO2-Zertifikate kaufen soll,
wird ein Ende der Kohle in Deutschland lediglich den Kohlestrom in Polen günstiger machen.
Daher lieber kein Kohleausstieg in Deutschland.
Dass es sich hierbei nur um ein Spiel auf Zeit handelt wird nicht erwähnt.

Der Höhepunkt der Disskussion wird erreicht indem auf den Eingriff von seitens des Staates fokussiert wird.
Der frühzeitige Kohleausstieg und die Solidarität mit den Bewohnern der umliegenden Dörfer sei der größte Schwachsinn.
Die Bewohner des Dorfes Holzweiler, welches ursprünglich dem Tagebau weichen sollte, wären inzwischen gerne umgezogen.
Der Grund: Die Bewohner des Dorfes sind dann in nächster Nähe zur Kohlegrube, und \enquote{keiner will so ein Loch vor der Haustür haben}.
Ein wenig Recherche ergibt: Die umliegenden Dörfer weiterhin abgebaggert, Wirtshäuser und Tankstellen dürfen mit einem
Umsatzeinbruch rechnen \footnote{\url{https://www.rundschau-online.de/region/rhein-erft/garzweiler-ii-holzweiler-bleibt-3079860}}.
Daher zeigt sich RWE solidarisch mit den Dorfbewohnern und regt alle Mitarbeiter dazu an, bei der Gegendemo der Anti-Umsiedlungs-Demonstration
teilzunehmen. Damit solle der Instrumentalisierung der Dorfbewohner von seiten der Medien entgegen gewirkt werden.

Außerdem zeige sich zum Beispiel in Hambach: Umweltschützer fällen dort Bäume während der Brutzeit um weiterhin
Holz für die Waldbesetzung zu haben.
Der Kommentar, dass die Vögel besser in der entstehenden Kohlegrube hätten brüten können überdeckt den Antwortversuch
mithilfe von Umsiedlung der einheimischen Tiere werde die vorhandene Natur geschützt.

Alles in allem bringt der Kohleausstieg in 2037 statt 2045 wohl einiges bei RWE durcheinander.
Und das obwohl \enquote{7 Jahre doch nicht die Welt retten} \footnote{Die UN schlägt vor, den weltweiten CO2 Ausstoß ab 2020 nicht weiter zu steigern, ab dort sogar fallen zu lassen (\url{https://www.theguardian.com/environment/2014/nov/02/rapid-carbon-emission-cuts-severe-impact-climate-change-ipcc-report})}.

Abschließend noch Kommentare zum Engagement von RWE.
Die durch die 30 Meter breiten Kohleflöze fehlende Masse in Garzweiler wird nach Ende des Kohleabbaus durch einen Restsee aufgefüllt.
Damit entsteht in Garzweiler ab 2035 wohl der größte innerdeutsche See.
Außerdem werden abgerissene Autobahnen und Siedlungen wieder aufgebaut, abgetragene Fläche renaturiert, 
alles auf Kosten von RWE.
Nach 10 Jahren sei die ehemalige Fläche des Tagebaus auch wieder bebaubar.
Es werden Fotos gezeigt, von einem Dorf, dahinter eine Kohlegrube.
Daneben dasselbe Dorf, dahinter das Ergebnis der Renaturierungsarbeiten von RWE, eine saftig grüne Landschaft.
Ein Bild von vor dem Loch fehlt, es wird jedoch gemunkelt, dass die Rheinisch-Westfälische Landschaft neben trockenen
Feldern vorher nicht viel zu bieten hatte.

Damit sind wir fertig und begeben uns zurück zum Bus.
Manche sind unzufrieden mit dem Vortrag. Man kommt sich nicht ernst genommen vor.
Die teils theatralischen Ausführungen der Vortragenden haben nicht geholfen.
Es geht auf zur Bustour um die Grube.

Neben der Straße liegt bereits rekultivierte Landschaft.
Die Hänge in Richtung Grube und die daneben liegenden Flächen sind bewachsen,
unterschiedliche Vegetation wurde hier gesetzt um den unterschiedlichen Ansprüchen gerecht zu werden.
Der Bus trifft auf eine Verladestation.
Hier werden bis zu 100 Tonnen Kohle pro Waggon auf den Zug mit 14 Wägen aufgeladen,
der die Kohle zu wenig weiter entfernten Kraftwerken transportiert.
Die Gleise hierfür sind teuer, der Unterboden muss verstärkt werden, da das Gewicht
des Zuges gewöhnliche Bahnschienen zerstören würde.
Was nicht per Zug transportiert wird wandert auf gut zwei Meter breiten Endlosbändern
an denen wir jetzt entlangfahren.
Sie werden auch für den gesamten grubeninternen Transport genutzt, um die Kohle aus dem Loch
im sogenannten Kohlebunker zu sammeln.

Große Absetzer werfen die Kohle sanft auf die sich dort bildenden schwarzen Haufen.
Weiter fahren wir entlang der Transportbänder, minutenlang.
Neben Kohle wird hier auch Kies befördert, es ist auch ein kleines Geschäftsfeld von RWE
und wird als Baustoff verwertet.

An einem Aussichtspunkt angelangt stehen wir vor einem Schaufelradbagger.
Er ist der kleinste Bagger von RWE mit lediglich 3200 Tonnen Gewicht.
Die Bagger in der Grube wiegen bis zu 13400 Tonnen, davon 40 Tonnen Rostschutzfarbe.
Wenn Baggerfahrer eine solche Maschine bedienen müssen sie sich alle zwei Stunden abwechseln,
da sonst die ständige Überwachung der Technik zu anstrengend wird.
Wie alles in der Grube, mit Ausnahme von kleinen PKW, wird er mit Strom betrieben,
dem Strom, der aus der hier geförderten Braunkohle gewonnen wird.
Auch dafür gibt es Verlängerungskabel mit Steckdosen.
Sie haben etwa die Größe eines Fiat.

An dem Kohleband findet sich außerdem eine Analysestelle.
Sie stellt die Qualität der geförderten Kohle fest, um ihr eine Stelle im Kohlebunker
zuzuweisen.

Dann öffnet sich vor uns die Grube. Zehn bis fünfzehn der riesigen Bändern
wandern dort hinein, nebeneinander, in Richtung der riesigen Kohleförderer die von
hier kaum sichtbar sind.
Auf der linken Seite, der Abtragungsseite, sind deutlich sichtbar die Sedimentschichten
zu erkennen.
Darunter auch drei schwarze Schichten, die 30 Meter breiten Kohleflöze.
Auf der anderen Seite wird der Abraum wieder abgesetzt.
Sie zeichnet sich durch deutlich fehlende Struktur aus,
die Sedimentschichten werden hier wild durcheinander gemischt abgelegt.

Für jede Tonne Kohle die hier gefördert wird, werden zwischen 3 und 5 Tonnen Gestein
abgetragen.
Wir fahren um die Grube herum an verrammelten, verlassenen Dörfern vorbei.
Wo man auch hinschaut sieht man die kleinen grünen Grundwasserpumpen.
Gelbe Bagger verschwinden im Anblick neben ihren großen, dunkelbraunen Schaufelradbrüdern.
Ein enormer Aufwand der sich im Anteil des Braunkohlestroms im deutschen Strommixes wiederfindet.
Etwa ein Viertel des Stroms in Deutschland kommt aus der Braunkohle.

Dennoch, auf die Klimafolgen des Kohlestroms angesprochen muss und die RWE Mitarbeiterin letztlich beipflichten.
\enquote{Der Kohleausstieg hat zu spät angefangen. Da bin ich ganz ihrer Meinung.}
Ein kleiner Erfolg.


\end{document}