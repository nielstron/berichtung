\documentclass{article}
\RequirePackage[utf8]{inputenc}
\RequirePackage[T1]{fontenc}
\RequirePackage[ngerman]{babel} % TODO: Make this an option
\RequirePackage[babel,german=quotes]{csquotes}
\RequirePackage{hyperref}

\title{Ausflug in die Kohlegrube}
\author{Niels Mündler}
\date{\today}

\begin{document}
\maketitle

Die Exkursion in den Braunkohletagebau wurde mit großem Andrang erwartet.
Lediglich limitiert durch die Größe des Busses fuhr so etwa die halbe Akademie 14 Kilometer in den Süden Richtung Garzweiler.

Der Tagebau kündigte sich ab und an bereits weit vorher durch auf den Feldern verstreute kleine grüne Pumpen an.
Diese sorgen dafür, dass der Grundwasserspiegel von ca. 10 Meter auf die unter 230 Meter gesenkt wird, die der Tagebau Garzweiler tief ist.
Eine RWE-Mitarbeiterin wird uns später erzählen, dies habe keine Auswirkungen auf die umliegende Landwirtschaft, die weitreichende Folgen für Wasserqualität und grundwasserabhängige Natur wurden jedoch verschwiegen
\footnote{\url{https://www.bund-nrw.de/themen/mensch-umwelt/braunkohle/hintergruende-und-publikationen/braunkohle-und-umwelt/braunkohle-und-wasser/}}

Ohne große Ankündigung öffnet sich plötzlich zu unserer linken ein gigantisches Loch.
Gespickt mit kleinen Maschinen fahren wir entlang der Kohlegrube.
Zur Aussicht gehören hier: Kohlekraftwerke, mit Wasserdampfsäulen die aussehen als wären sie für die Wolken am Himmel verantwortlich.
Zehn Meter hohe Rechen, die umgekehrt im Boden stecken und eigentlich Sprühnebel erzeugen um aufsteigen Stäube zu binden.
Windräder, wild verstreut auf Hügeln, in nächster Nähe zu ihrem dicken Onkel der Energieerzeugungsfamilie.

Schließlich kommen wir am RWE Informationszentrum an.
Davor steht eine ausrangierte Kohleradschaufel wie eine Skulptur.
Weiter hinten wurden andere Schaufeln bunt angemalt und sollen für Kinder Spielmöglichkeiten bieten.


\end{document}