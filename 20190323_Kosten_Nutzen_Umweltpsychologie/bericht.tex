\documentclass{article}
\RequirePackage[utf8]{inputenc}
\RequirePackage[T1]{fontenc}
\RequirePackage[ngerman]{babel} % TODO: Make this an option
\RequirePackage[babel,german=quotes]{csquotes}
\RequirePackage{hyperref}

\title{Ausflug in die Kohlegrube}
\author{Niels Mündler}
\date{\today}

\begin{document}
\maketitle

Als Individuum ist auch stets wichtig, welchen persönlichen Nutzen man aus einer Handlung zieht.
Dabei muss dieser Nutzen oder Gewinn nicht ökonomischer Natur sein.
Es kann sich durchaus auch um das Vermeiden einer Bestrafung oder das Erhalten einer Belohnung handeln.
Eine Belohnung kann materiell sein (z.B. Schokolade), aber
auch eine lustige oder ästethische Darbietung, welche im Individuum positive Gefühle auslöst, kann ausreichen.

Studien zeigen dass dabei die meisten etwas intelligenteren Lebewesen (darunter auch Menschen) besser auf 
eine Belohnung ansprechen als auf Bestrafung.
Zudem ist wichtig, dass die Belohnung zeitnah und in angemessener Größe erhalten wird.
Ist sie zu groß, kann sie Skepsis auslösen und zudem die intrinsische Motivation überdecken (die stärkste aller Motivationen, aus dem Individuum selbst heraus).

Ein gutes Beispiel hierfür sind Mülleimer, die ein Geräusch machen das so klingt als würde ein Gegenstand einen
tiefen Schacht herunterfallen und damit lustig wirken oder Treppen, die Klaviernoten spielen, wenn man sie besteigt.
Sie belohnen das Gehirn direkt und in kleinem Maße sodass Menschen nachhaltig zum Müll korrekt entsorgen oder 
Treppe statt Rolltreppe benutzen motiviert werden.

Auch für selbstgesteckte Ziele ergibt sich daher die wichtige Komponente, dass konkrete Vorteile einer ökologischen Handlungsweise
ausgemacht werden sollten.
Selbst wenn es diese direkt nicht gibt, können konkrete Meilensteine gesteckt werden sollten sodass sich das Gehirn bei erreichen
eines solchen Teilzieles selbst mit einem Erfolgsgefühl belohnen kann.

\end{document}